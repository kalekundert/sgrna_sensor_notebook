\documentclass{article}

\usepackage{graphicx}

\newcommand{\subcap}[1]{\textbf{#1:}}
\newcommand{\rxb}[1]{rxb 11,1}

\begin{document}

\begin{figure}
 \centering
 \includegraphics[width=0.7\textwidth]{rxb_11_1_mechanism}
 \caption{The proposed mechanism for \rxb{}.  We believe that the indicated 
 uracil must be unpaired in order for the sgRNA to bind Cas9.  Our hypothesis 
 concerning \rxb{} is that when no ligand is present, the base of the aptamer 
 comes apart and allows the top of the stem to melt, which allows the indicated 
 uracil to interact productively with Cas9.  When ligand is present, the base 
 of the aptamer comes together, locks the stem in place, and sequesters the 
 uracil in a base pair.}
\end{figure}

\begin{figure}
 \centering
 \includegraphics[width=0.7\textwidth]{20170315_test_w11_m11_figure}
 \caption{Mutational analysis of \rxb{}. \subcap{Left panels} cell populations 
 in the presence (solid lines) and absence (dashed lines) of theophylline.  We 
 used flow cytometry to measure fluorescence values for individual cells, and 
 we used Gaussian kernel density estimation (KDE) to calculate population 
 distributions.  The x-axis shows GFP fluorescence divided by RFP fluorescence.  
 The sgRNAs target GFP, and RFP is being expressed as an internal control for 
 cell size.  The modes of the populations are indicated by the plus-signs below 
 the plots.  \subcap{Right panels} ligand-induced fold changes in fluorescence.  
 Specifically, fold change is the ratio of the modes of the apo and holo 
 populations for the sensor in question.  \subcap{A} the positive control, the 
 negative control (an sgRNA with the 5' side of the nexus stem mutated to CC), 
 and \rxb{}.  \subcap{B} variants of \rxb{} where the highlighted base pairs 
 have been strand swapped.  \subcap{C} variants of \rxb{} that attempt to 
 modulate sensor activity by either weakening or strengthening the stem above 
 the uracil that is believed to flip out to interact with Cas9.  The variants 
 are sorted from weakest to strongest.  \subcap{Labels} the uracil believed to 
 be important for Cas9 binding is colored red.  GC/CG mutations are highlighted 
 in black, AU/UA mutations in dark grey, and GU/UG mutations in light grey.}
\end{figure}

We expected that all the strand-swapped variants of \rxb{} -- other than the 
third one that moved the uracil -- would still react to the presence of 
theophylline (Fig 2b).  Every position but the first met our expectations.  The 
first position, however, was especially surprising because Briner et al.  
(2014) didn't see a significant effect when they replaced it with an AU base 
bair.  It may be that that position is relevant to the mechanism of \rxb{} in 
a way we don't yet understand.

The variants of \rxb{} that modulate the strength of the stem above the 
uracil have the expected effect on sensor function (Fig 2c).  The constructs 
towards the top have weaker stems, which means the uracil should spend more 
time unpaired and the Cas9 complex should form more readily.  This should lead 
to better CRISPRi repression, which is what we observe.  The constructs towards 
the bottom have stronger stems, and the reverse logic applies.  Looking at the 
data as a whole, the traces exhibit a striking diagonal pattern.  The third 
construct in particular may be useful in it's own right.  Although it's fold 
change is slightly worse than that of \rxb{}, it turns on more completely, and 
that may be useful for some applications.

\end{document}
